\documentclass[]{article}

%opening
\title{Reflexionsbericht Orientierung Raumplanung}
\author{Ida Hönigmann}

\begin{document}

\maketitle

\begin{abstract}

\end{abstract}

\section{Interesse an der Raumplanung und Erwartungen}
Die letzten sechs Jahre habe ich mich mit Informatik und Mathematik beschäftigt. Vor allem nach zwei Semestern Technische Mathematik Studium an der TU Wien, wollte ich etwas ausprobieren, das näher am tatsächlichen Leben ist. Obwohl ich auch weiterhin vorwiegend Technische Mathematik studiere, will ich versuchen mir Wissen anzueignen, mit dem ich hoffentlich meine Umwelt ein Stückchen lebenswerter gestalten kann.

Dabei ist es sicher möglich diese verschiedenen Bereiche zusammenzuführen. Dabei denke ich zum Beispiel an Modelle und Simulationen von Verkehrssystemen oder ähnliches.

Die Orientierungs-Veranstaltung hat mir bewusst gemacht, wie vielseitig Raumplanung ist. Zuvor habe ich bei Raumplanung an das Aufstellen von Parkbänken und die Planung von Straßenzügen gedacht. Jetzt weiß ich, wie entscheidend die politischen Aspekte bei der Umsetzung der raumplanerischen Vision ist.

\section{Politik in der Raumplanung?}
Obwohl man nach der Orientierungs-Veranstaltung wahrscheinlich noch nicht sehr tiefes Wissen über Methoden zur raumplanerischen Gestaltung erhalten hat, habe ich das Gefühl, dass wesentliche Gestaltungsmöglichkeiten nur durch von der Politik beschlossenen Maßnahmen möglich sind. So scheint zum Beispiel der Widmungs- und Bebauungsplan ein wesentliches Werkzeug auf Gemeindeebene zu sein.

Die doch große Rolle der Politik in der Raumplanung hat mich überrascht, da es sich (fast) nie um die Kernkompetenz der entscheidenden Politiker handelt. Daher hätte ich erwartet, dass man versucht die Gestaltung des Raumes, auch wegen seinem langen zeitlichen Horizont, von der sich verhältnismäßig schnell wechselnden Politik trennt. Hoffentlich gibt es die Möglichkeit die Vor- und Nachteile beider Weisen in den Lehrveranstaltungen der Raumplanung zu diskutieren.

\section{Kommunikation}
Eine Fähigkeit, die ich hoffe durch den Besuch von Lehrveranstaltungen der Raumplanung persönlich aufbessern zu können ist die Kommunikationsbereitschaft. Die Vortragenden aus dem PraktikerInnenforum haben gezeigt, dass Sie in Ihrem Beruf nicht nur andere Personen von Ihren Ideen überzeugen müssen, sondern andererseits auch die Wünsche von Stakeholdern durch aufmerksames Zuhören erfassen müssen. Beide Bereiche sind sicherlich etwas bei dem ich etwas dazulernen kann.

\section{Eindrücke der Exkursion und Workshops}
\subsection{Workshop 1: Raum - Planung?}
Es war sehr interessant eine Herangehensweise, die sehr ähnlich einer mathematischen war, an das Thema Raumplanung zu sehen. Wie in jeder Mathematik-Vorlesung üblich, haben wir begonnen die einzelnen Begriffe, insbesondere Raum und Planung, zu definieren oder zumindest versucht sie genauer zu erklären.

Raum ist, wie mir durch den Workshop klar wurde, ein Begriff, den man nur schwierig beschreiben kann. Ob man an den geografischen, physikalischen oder mathematischen Begriff Raum denkt, in der Raumplanung meint man doch immer etwas anderes und alles drei zugleich.

Einfacher zu beschreiben war das Wort Planung. Ziel, Information, Alternativen, Entscheidung, Durchführung, Finanzen, Zeitplan waren die sieben entscheidenden Punkte, die Planung laut Workshop ausmachen. Dadurch unterscheidet sich das Verhalten von Tieren und dem von Menschen unter anderem durch die Fähigkeit zu planen.

\subsection{Workshop 2: Einkaufszentren}
Der Zweck des Workshops war es typische Problemstellungen der Raumplanung am Beispiel Einkaufszentrum zu diskutieren.

Schon bei der Planung des Ortes eines neuen Einkaufszentrum entstehen meist Konflikte. So vertreten die Gemeinden das Interesse, dass das Einkaufszentrum möglichst auf ihrem Gemeindegebiet gebaut werden soll. So entstehen mit einem Schlag viele Arbeitsplätze in der entsprechenden Gemeinde, was wiederum zu höheren Steuereinnahmen führt. Da alle Gemeinden in der Region diese Interessen vertreten ergibt sich meist eine Wettbewerbssituation zwischen den verschiedenen Gemeinden.

Weiter verschärft wird die Problematik durch die Erfahrenheit des verhandelnden Teams auf Seiten des Investors, die oft in starkem Kontrast zu jener auf Seiten der Gemeinde steht. Dahingehend wäre eine Instanz auf höherer Ebene, beispielsweise bundeslandweit, die Koordination der Standorte von Einkaufszentren regelt, möglicherweise eine gute Idee.

Eine der vielen Auswirkungen von Einkaufszentren auf eine Stadt oder ein Dorf ist die Zunahme von Leerstand in den Innenstädten. Einige Ideen, die dem entgegenwirken können, haben wir in kleinen Gruppen diskutiert und präsentiert, so zum Beispiel Maßnahmen zum Erhalt von kleinen Läden (z.B. durch Förderungen des Mietpreises) oder attraktive Gestaltung der Innenstädte als Shopping-Möglichkeit (z.B. durch Begegnungszonen). Keine dieser Optionen alleine, wird eine große Veränderung bringen. Somit bleibt die Problematik der ''leeren Innenstadt'' eine komplexe, ungelöste Problemstellung.

Mir von diesem Workshop sehr in Erinnerung geblieben sind die Mittel mit denen Betreiber von Geschäften in Einkaufszentren geltende Bestimmungen umgehen. Wie wir gelernt haben sind besonders Lebensmittelgeschäfte in Einkaufszentren sehr stark geregelt. Einige Einschränkungen treten erst ab einer gewissen Quadratmeterzahl der Geschäftsfläche in Kraft. So wird zum Beispiel die Getränkeabteilung nicht zur Geschäftsfläche gezählt, da diese als Lager gilt. Ich habe den Verdacht, dass genau diese Taktik bei einem Supermarkt in der Nähe meines Wohnorts angewandt wird. Dort ist genau diese Abteilung nicht, wie sonst alles, ordentlich angeschrieben und mit Werbung versehen, sondern ein Regallager.

\section{Zeitungsartikel und eigene Meinung}
\subsection{Öffentlicher Raum: Warum unsere Stadtplanung diskriminierend ist}
Der Artikel behandelt eine interessante Fragestellung: Soll man einen Ort, der zurzeit von bestimmten Personengruppen genutzt wird, für diese optimieren oder versuchen andere Personengruppen dort zusätzlich anzusiedeln?



\subsection{Warum ein Immobilienprojekt in einer steirischen Gemeinde für Empörung sorgt}

\subsection{Diskussion über Flächenfraß: Bauen bis ins Bodenlose}
Verknüpfung Workshop 2

\subsection{Trend in der Stadtplanung: Was an Superblocks super sein soll}
Verknüpfung Exkursion

\subsection{Raumplanung: Der harte Kampf um die Innenstädte}
Verknüpfung Workshop 2

\section{Studienplan - Wie geht es weiter?}
Nachdem ich nicht plane, das Bachelorstudium Raumplanung zu absolvieren, sondern aus Interesse einzelne Lehrveranstaltungen besuchen möchte, ist meine Sichtweise zum Studienplan, die eines Außenstehenden. Folgende fünf Aspekte sind mir beim Durchsehen des Studienplans aufgefallen:

\subsection{Einstieg ins Studium}
Was mir positiv auffällt ist der praxisorientierte Einstieg in das Studium. Da das Modul der Studieneingans- und Orientierungsphase aus zwei Veranstaltungen besteht, die sich beide mit Raumplanung im Gesamten und nicht nur einem kleinen Teilbereich befassen, ist die StEOP sicherlich für alle Studierende im ersten Semester des Raumplanungsstudiums interessant. Zusätzlich erfährt man schon sehr früh mit welchen Tätigkeiten Raumplaner beschäftigt sind und kann so besser entscheiden ob einen die Tätigkeit eines Raumplaners anspricht.

\subsection{Lehrveranstaltungstypen}
Im Vergleich mit dem Bachelorstudium Technische Mathematik fällt auf, dass keine Lehrveranstaltung als reine Übung (UE) organisiert ist. Vielleicht wurde stattdessen der Typ Vorlesung mit integrierter Übung (VU) gewählt, da diese Lehrveranstaltungen durchschnittlich nur ungefähr drei ECTS Aufwand bedeuten oder um Anwesenheitspflicht zu ermöglichen.

\subsection{Freies Wahlfach im ersten Semester}
Die Entscheidung im ersten Semester bereits ein Freies Wahlfach oder ein Transferable Skill vorzusehen, halte ich für nicht sehr gelungen, da sich aus meiner Erfahrung die meisten erstsemestrigen Studierenden noch nicht genug mit der Organisation einer Universität auskennen um sich rechtzeitig für solch eine Lehrveranstaltung anzumelden.

\subsection{Informatik im Raumplanungsstudium}
Vielleicht auch wegen meiner Ausbildung als Informatikerin kommt mir das Thema Informatik im Studienplan zu schnell und kurz abgewickelt vor. Einführung in das Programmieren für Raumplaner, also eine Lehrveranstaltung, die Grundlagen des Programmieren lehrt, würde es den Studierenden zum Beispiel ermöglichen, komplexe Zusammenhänge in raumplanerisch relevanten Informationen, die nicht mittels SQL Abfragen, wie sie in Datenbanken und Informationsmanagement beigebracht werden, zu erforschen. Beispiele in diese Richtung, die ich interessant fände, wäre zum Beispiel eine Hauptkomponentenanalyse (Principal component analysis) um den Mietpreise in neuen Stadtteilen oder die gefühlte Außentemperatur, abhängig von unterschiedlichsten Faktoren zu schätzen. Selbst wenn sich diese Beispiele vielleicht nicht in einer Vorlesung in einem Semester ausgehen, wäre es doch bereits hilfreich in Ansätzen zu verstehen, wie die Software, mit der man arbeitet, funktioniert.

\subsection{Insgesamte Gestaltung}
Insgesamt ist es meiner Ansicht nach im neuen Studienplan gelungen alle wesentlichen Teilbereiche der Raumplanung abzudecken und jeweils ein Grundwissen zu etablieren. Ich persönlich werde auf jeden Fall auch in den nächsten Semestern immer wieder einzelne Vorlesungen aus dem Bachelorstudium Raumplanung besuchen.

\end{document}
