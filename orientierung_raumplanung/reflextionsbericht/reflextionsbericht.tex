\documentclass[]{article}

%opening
\title{Reflexionsbericht Orientierung Raumplanung}
\author{Ida Hönigmann}

\begin{document}

\maketitle

\begin{abstract}

\end{abstract}

\section{Interesse an der Raumplanung und Erwartungen}
Die letzten sechs Jahre habe ich mich mit Informatik und Mathematik beschäftigt. Vor allem nach zwei Semestern Technische Mathematik Studium an der TU Wien, wollte ich etwas ausprobieren, das näher am tatsächlichen Leben ist. Obwohl ich auch weiterhin vorwiegend Technische Mathematik studiere, will ich versuchen mir Wissen anzueignen, mit dem ich hoffentlich meine Umwelt ein Stückchen lebenswerter gestalten kann.

Dabei ist es sicher möglich diese verschiedenen Bereiche zusammenzuführen. Dabei denke ich zum Beispiel an Modelle und Simulationen von Verkehrssystemen oder ähnliches.

Die Orientierungs-Veranstaltung hat mir bewusst gemacht, wie vielseitig Raumplanung ist. Zuvor habe ich bei Raumplanung an das Aufstellen von Parkbänken und die Planung von Straßenzügen gedacht. Jetzt weiß ich, wie entscheidend die politischen Aspekte bei der Umsetzung der raumplanerischen Vision ist.

\section{Politik in der Raumplanung?}
Obwohl man nach der Orientierungs-Veranstaltung wahrscheinlich noch nicht sehr tiefes Wissen über Methoden zur raumplanerischen Gestaltung erhalten hat, habe ich das Gefühl, dass wesentliche Gestaltungsmöglichkeiten nur durch von der Politik beschlossenen Maßnahmen möglich sind. So scheint zum Beispiel der Widmungs- und Bebauungsplan ein wesentliches Werkzeug auf Gemeindeebene zu sein.

Die doch große Rolle der Politik in der Raumplanung hat mich überrascht, da es sich (fast) nie um die Kernkompetenz der entscheidenden Politiker handelt. Daher hätte ich erwartet, dass man versucht die Gestaltung des Raumes, auch wegen seinem langen zeitlichen Horizont, von der sich verhältnismäßig schnell wechselnden Politik trennt. Hoffentlich gibt es die Möglichkeit die Vor- und Nachteile beider Weisen in den Lehrveranstaltungen der Raumplanung zu diskutieren.

\section{Kommunikation}
Eine Fähigkeit, die ich hoffe durch den Besuch von Lehrveranstaltungen der Raumplanung persönlich aufbessern zu können ist die Kommunikationsbereitschaft. Die Vortragenden aus dem PraktikerInnenforum haben gezeigt, dass Sie in Ihrem Beruf nicht nur andere Personen von Ihren Ideen überzeugen müssen, sondern andererseits auch die Wünsche von Stakeholdern durch aufmerksames Zuhören erfassen müssen. Beide Bereiche sind sicherlich etwas bei dem ich etwas dazulernen kann.

\section{Eindrücke der Exkursion und Workshops}

\section{Zeitungsartikel und eigene Meinung}
\subsection{Öffentlicher Raum: Warum unsere Stadtplanung diskriminierend ist}
Fragestellung: Soll man einen Ort, der zurzeit von bestimmten Personengruppen genutzt wird, für diese optimieren oder versuchen andere Personengruppen dort zusätzlich anzusiedeln?

\subsection{Warum ein Immobilienprojekt in einer steirischen Gemeinde für Empörung sorgt}

\subsection{Diskussion über Flächenfraß: Bauen bis ins Bodenlose}

\subsection{Trend in der Stadtplanung: Was an Superblocks super sein soll}

\subsection{Raumplanung: Der harte Kampf um die Innenstädte}

\section{Studienplan - Wie geht es weiter?}

\end{document}
